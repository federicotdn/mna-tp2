	\documentclass[12pt,a4paper]{article}

\usepackage[utf8]{inputenc}
\usepackage{amsmath}
\usepackage{amsfonts}
\usepackage{makeidx}
\usepackage{amssymb}
\usepackage{hyperref}

\title{TPE Métodos Numéricos Avanzados\\Speech Compression}
\author{Federico Tedin - 53048\\Javier Fraire - 53023}

\makeindex

\begin{document}
\maketitle

\renewcommand{\abstractname}{Resumen:}
\begin{abstract}

\centering
Compresión de habla utilizando la FFT (Fast Fourier Transform) y codificación Huffman, en archivos WAVE.

\end{abstract}

\clearpage
\tableofcontents
\clearpage

\section{Introducción}
En éste trabajo práctico se muestra como es posible comprimir un archivo conteniendo datos de sonido. El fin es producir un archivo más pequeño que contenga aproximadamente la misma información sonora, teniendo en cuenta siempre la relación entre el tamaño final y el inicial.  En éste caso, todas las pruebas realizadas se hicieron sobre distintas muestras de voces humanas.  La herramienta utilizada para realizar este trabajo fue el idioma de programación Python 3.4, junto a algunas funciones y librerías externas.

\section{Metodología}

Para comprimir un archivo de audio, primero es necesario grabar algún sonido utilizando un \emph{software} de grabado de sonido (para éste trabajo, se decidió utilizar \emph{Audacity}).  De acuerdo con el enunciado, se debe utilizar un formato que no utilize compresión, y que tenga un sólo canal.  Se decidió utilizar el formato WAVE \cite{wave}, ya que tiene una estructura simple, y porque ya existen librerías de Python que permiten la lectura y escritura de éstos archivos.

	Los archivos WAVE (extensión \emph{.wav}) almacenan un valor (‘‘sample’’) de sonido por cada intervalo de tiempo $T$, donde $T$ es recíproco a la frecuencia de muestreo $f$.  En éste trabajo, se tomó $f = 8000 Hz$.  Los archivos utilizados en éste trabajo utilizan samples de tipo \emph{PCM} de 16 bits, con signo.  Entonces, un archivo de audio de $1$ segundo tendría exactamente $8000$ muestras, lo cual resultaría en un tamaño total de $8000 * 16 = 128000\: bits = 16000\:	bytes$.

	Los métodos utilizados estan basados mayormente en los que se explican en \cite{rajesh}, y en el enunciado del problema.

\clearpage

\begin{thebibliography}{9}
  
\bibitem{wave}
  Scott Wilson,
  WAVE PCM soundfile format,
  \url{https://ccrma.stanford.edu/courses/422/projects/WaveFormat/}

\bibitem{rajesh}
  G. Rajesh, A. Kumar and K. Ranjeet,
  \emph{Speech Compression using Different Transform Techniques}.
  Indian Institute of Information Technology, Design \& Manufacturing Jabalpur,

\end{thebibliography}

\end{document}