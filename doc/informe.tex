\documentclass[12pt,a4paper]{article}

\usepackage[utf8]{inputenc}
\usepackage{amsmath}
\usepackage{amsfonts}
\usepackage{makeidx}
\usepackage{amssymb}

\title{TPE Métodos Numéricos Avanzados\\Speech Compression}
\author{Federico Tedin - 53048\\Javier Fraire - 53023}

\makeindex

\begin{document}
\maketitle

\begin{abstract}

\centering
Compresión de habla utilizando la FFT (Fast Fourier Transform) y codificación Huffman, en archivos .WAV.

\end{abstract}

\clearpage
\tableofcontents
\clearpage

\section{Introducción}
En éste trabajo práctico se muestra como es posible comprimir un archivo conteniendo datos de sonido. El fin es producir un archivo más pequeño que contenga aproximadamente la misma información sonora, teniendo en cuenta siempre la relación entre el tamaño final y el inicial.  En éste caso, todas las pruebas realizadas se hicieron sobre distintas muestras de voces humanas.  La herramienta utilizada para realizar este trabajo fue el idioma de programación Python 3.4, junto a algunas funciones y librerías externas.

\section{Metodología}

adadf

\end{document}